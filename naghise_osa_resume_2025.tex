\documentclass[10pt, letterpaper]{article}

% Packages:
\usepackage[
    ignoreheadfoot, % set margins without considering header and footer
    top=2 cm, % seperation between body and page edge from the top
    bottom=2 cm, % seperation between body and page edge from the bottom
    left=2 cm, % seperation between body and page edge from the left
    right=2 cm, % seperation between body and page edge from the right
    footskip=1.0 cm, % seperation between body and footer
    % showframe % for debugging 
]{geometry} % for adjusting page geometry
\usepackage{titlesec} % for customizing section titles
\usepackage{tabularx} % for making tables with fixed width columns
\usepackage{array} % tabularx requires this
\usepackage[dvipsnames]{xcolor} % for coloring text
\definecolor{primaryColor}{RGB}{0, 0, 0} % define primary color
\usepackage{enumitem} % for customizing lists
\usepackage{fontawesome5} % for using icons
\usepackage{amsmath} % for math
\usepackage[
    pdftitle={Osa Naghise Resume},
    pdfauthor={Osa Naghise},
    colorlinks=true,
    urlcolor=primaryColor
]{hyperref} % for links, metadata and bookmarks
\usepackage[pscoord]{eso-pic} % for floating text on the page
\usepackage{calc} % for calculating lengths
\usepackage{bookmark} % for bookmarks
\usepackage{lastpage} % for getting the total number of pages
\usepackage{changepage} % for one column entries (adjustwidth environment)
\usepackage{paracol} % for two and three column entries
\usepackage{ifthen} % for conditional statements
\usepackage{needspace} % for avoiding page brake right after the section title
\usepackage{iftex} % check if engine is pdflatex, xetex or luatex

% Ensure that generate pdf is machine readable/ATS parsable:
\ifPDFTeX
    \input{glyphtounicode}
    \pdfgentounicode=1
    \usepackage[T1]{fontenc}
    \usepackage[utf8]{inputenc}
    \usepackage{lmodern}
\fi

\usepackage{charter}

% Some settings:
\raggedright
\AtBeginEnvironment{adjustwidth}{\partopsep0pt} % remove space before adjustwidth environment
\pagestyle{empty} % no header or footer
\setcounter{secnumdepth}{0} % no section numbering
\setlength{\parindent}{0pt} % no indentation
\setlength{\topskip}{0pt} % no top skip
\setlength{\columnsep}{0.15cm} % set column seperation
\pagenumbering{gobble} % no page numbering

\titleformat{\section}{\needspace{4\baselineskip}\bfseries\large}{}{0pt}{}[\vspace{1pt}\titlerule]

\titlespacing{\section}{
    % left space:
    -1pt
}{
    % top space:
    0.3 cm
}{
    % bottom space:
    0.2 cm
} % section title spacing

\renewcommand\labelitemi{$\vcenter{\hbox{\small$\bullet$}}$} % custom bullet points
\newenvironment{highlights}{
    \begin{itemize}[
        topsep=0.10 cm,
        parsep=0.10 cm,
        partopsep=0pt,
        itemsep=0pt,
        leftmargin=0 cm + 10pt
    ]
}{
    \end{itemize}
} % new environment for highlights


\newenvironment{highlightsforbulletentries}{
    \begin{itemize}[
        topsep=0.10 cm,
        parsep=0.10 cm,
        partopsep=0pt,
        itemsep=0pt,
        leftmargin=10pt
    ]
}{
    \end{itemize}
} % new environment for highlights for bullet entries

\newenvironment{onecolentry}{
    \begin{adjustwidth}{
        0 cm + 0.00001 cm
    }{
        0 cm + 0.00001 cm
    }
}{
    \end{adjustwidth}
} % new environment for one column entries

\newenvironment{twocolentry}[2][]{
    \onecolentry
    \def\secondColumn{#2}
    \setcolumnwidth{\fill, 4.5 cm}
    \begin{paracol}{2}
}{
    \switchcolumn \raggedleft \secondColumn
    \end{paracol}
    \endonecolentry
} % new environment for two column entries

\newenvironment{threecolentry}[3][]{
    \onecolentry
    \def\thirdColumn{#3}
    \setcolumnwidth{, \fill, 4.5 cm}
    \begin{paracol}{3}
    {\raggedright #2} \switchcolumn
}{
    \switchcolumn \raggedleft \thirdColumn
    \end{paracol}
    \endonecolentry
} % new environment for three column entries

\newenvironment{header}{
    \setlength{\topsep}{0pt}\par\kern\topsep\centering\linespread{1.5}
}{
    \par\kern\topsep
} % new environment for the header

\newcommand{\placelastupdatedtext}{% \placetextbox{<horizontal pos>}{<vertical pos>}{<stuff>}
  \AddToShipoutPictureFG*{% Add <stuff> to current page foreground
    \put(
        \LenToUnit{\paperwidth-2 cm-0 cm+0.05cm},
        \LenToUnit{\paperheight-1.0 cm}
    ){\vtop{{\null}\makebox[0pt][c]{
        \small\color{gray}\textit{Last updated in September 2025}\hspace{\widthof{Last updated in September 2025}}
    }}}%
  }%
}%

% save the original href command in a new command:
\let\hrefWithoutArrow\href

% new command for external links:
\begin{document}
\newcommand{\AND}{\unskip
	\cleaders\copy\ANDbox\hskip\wd\ANDbox
	\ignorespaces
}
\newsavebox\ANDbox
\sbox\ANDbox{$|$}

\begin{header}
	\fontsize{25 pt}{25 pt}\selectfont Osa Naghise - Software Engineer

	\vspace{5 pt}

	\normalsize
	\mbox{Orlando, Fl}%
	\kern 5.0 pt%
	\AND%
	\kern 5.0 pt%
	\mbox{\hrefWithoutArrow{mailto:naghiseosa2000@gmail.com}{naghiseosa2000@gmail.com}}%
	\kern 5.0 pt%
	\AND%
	\kern 5.0 pt%
	\mbox{\hrefWithoutArrow{https://www.linkedin.com/in/osa-naghise}{www.linkedin.com/in/osa-naghise}}%
	\kern 5.0 pt%
	\AND%
	\kern 5.0 pt%
	\mbox{\hrefWithoutArrow{https://github.com/Osa-nag00}{https://github.com/Osa-nag00}}%
\end{header}

\vspace{6 pt - 0.3 cm}


\vspace{0.2 cm}


% Experience section
\section{Experience}
\begin{twocolentry}{
		Aug 2024 – Present
	}
	\textbf{Solution Analyst II}, Deloitte -- Lake Mary, FL\end{twocolentry}

\vspace{0.10 cm}
\begin{onecolentry}
	\begin{highlights}
		\item Developed an AWS Lambda function triggered by S3 uploads to automate
		ingestion of CSV data into a DynamoDB table
		\item Implemented automated validation and peer code review processes within
		CI/CD pipelines, improving deployment quality and reducing rollback
		incidents.
		\item Managed Salesforce orgs (scratch, sandbox, staging, prod), ensuring
		consistency across environments.
		\item Developed a TypeScript CLI tool using Salesforce CLI to manage test data
		lifecycle—automating the cleanup of stale records and bulk upsert of
		custom sObjects and test users across scratch and sandbox orgs, saving
		hours per test cycle.
		\item Created a CLI tool in TypeScript to parse deployment logs from Salesforce
		deployments and surface actionable alerts in Bitbucket pipelines,
		accelerating incident response.
		\item Migrated legacy Bash automation to TypeScript-based CLI tools, improving
		extensibility and maintainability through static typing, better tool support,
		and enhanced collaboration.
		\item Participated in Agile ceremonies, contributing to sprint planning,
		retrospectives, and daily stand-ups to ensure alignment with team goals.
		\item Developed and maintained comprehensive project documentation, facilitating knowledge transfer and onboarding for new team members.
		\item Designed and implemented monitoring dashboards using AWS CloudWatch and custom metrics, enabling proactive detection of system anomalies.
		\item Collaborated with cross-functional teams to define and refine requirements.
	\end{highlights}
\end{onecolentry}


\vspace{0.2 cm}

\begin{twocolentry}{
		Aug 2023 – Sep 2024
	}
	\textbf{Software Engineer I}, University Of Central Florida -- Orlando, FL\end{twocolentry}

\vspace{0.10 cm}
\begin{onecolentry}
	\begin{highlights}
		\item Extended an open-source SQLite project with custom features tailored to
		research team needs, improving functionality and supporting ongoing
		development.
		\item ntegrated Maven and Gradle into Java project workflows, enhancing build
		efficiency and enabling parallel execution to accelerate development
		cycles.
		\item Collaborated with a cross-functional team to modernize a legacy GUI
		application with JavaFX, improving responsiveness and maintainability.
		\item Implemented automated testing frameworks using JUnit and Mockito,
		increasing code coverage and reducing regression defects.
		\item Participated in code reviews and pair programming sessions, fostering a culture of continuous learning and quality improvement.
	\end{highlights}
\end{onecolentry}

% Education section
\section{Education}

\begin{onecolentry}
	\textbf{University of Central Florida}\\
	\textit{Bachelor of Science in Computer Science} \\
	\begin{tabular*}{\textwidth}{@{\extracolsep{\fill}} l r}
		\textbf{GPA: 3.6/4.0} & \textit{Aug 2021 -- May 2024} \\
	\end{tabular*}
\end{onecolentry}



\begin{onecolentry}
	\textbf{Pensacola State College}\\
	\textit{Associate of arts} \\
	\begin{tabular*}{\textwidth}{@{\extracolsep{\fill}} l r}
		\textbf{GPA: 3.8/4.0} & \textit{Aug 2019 -- May 2021} \\
	\end{tabular*}
\end{onecolentry}


% Technologies section
\section{Technologies}
\begin{onecolentry}
	\textbf{Languages:} C++ \textbar{} Java \textbar{} SQL \textbar{} JavaScript \textbar{} TypeScript \textbar{} Python \textbar{} HTML \textbar{} CSS \textbar{} Bash
\end{onecolentry}

\vspace{0.2 cm}

\begin{onecolentry}
	\textbf{Technologies:} AWS (Lambda, S3, DynamoDB, IAM, CloudWatch), Salesforce (SOQL, CLI, scratch orgs, sandbox management), Bitbucket Pipelines, Git, Maven, Gradle, JUnit, Mockito, JavaFX, SQLite, TypeScript, JavaScript, Node.js, Bash, CI/CD, Agile
\end{onecolentry}


\end{document}
